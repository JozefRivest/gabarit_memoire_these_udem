% Options for packages loaded elsewhere
% Options for packages loaded elsewhere
\PassOptionsToPackage{unicode}{hyperref}
\PassOptionsToPackage{hyphens}{url}
\PassOptionsToPackage{dvipsnames,svgnames,x11names}{xcolor}
%
\documentclass[
  french,
  12pt,
  letterpaper,
]{report}
\usepackage{xcolor}
\usepackage[margin=2.5cm,headsep=22pt,headheight=11pt,footskip=33pt,ignorehead,ignorefoot,heightrounded]{geometry}
\usepackage{amsmath,amssymb}
\setcounter{secnumdepth}{5}
\usepackage{iftex}
\ifPDFTeX
  \usepackage[T1]{fontenc}
  \usepackage[utf8]{inputenc}
  \usepackage{textcomp} % provide euro and other symbols
\else % if luatex or xetex
  \usepackage{unicode-math} % this also loads fontspec
  \defaultfontfeatures{Scale=MatchLowercase}
  \defaultfontfeatures[\rmfamily]{Ligatures=TeX,Scale=1}
\fi
\usepackage{lmodern}
\ifPDFTeX\else
  % xetex/luatex font selection
  \setmainfont[]{Times New Roman}
\fi
% Use upquote if available, for straight quotes in verbatim environments
\IfFileExists{upquote.sty}{\usepackage{upquote}}{}
\IfFileExists{microtype.sty}{% use microtype if available
  \usepackage[]{microtype}
  \UseMicrotypeSet[protrusion]{basicmath} % disable protrusion for tt fonts
}{}
\usepackage{setspace}
% Make \paragraph and \subparagraph free-standing
\makeatletter
\ifx\paragraph\undefined\else
  \let\oldparagraph\paragraph
  \renewcommand{\paragraph}{
    \@ifstar
      \xxxParagraphStar
      \xxxParagraphNoStar
  }
  \newcommand{\xxxParagraphStar}[1]{\oldparagraph*{#1}\mbox{}}
  \newcommand{\xxxParagraphNoStar}[1]{\oldparagraph{#1}\mbox{}}
\fi
\ifx\subparagraph\undefined\else
  \let\oldsubparagraph\subparagraph
  \renewcommand{\subparagraph}{
    \@ifstar
      \xxxSubParagraphStar
      \xxxSubParagraphNoStar
  }
  \newcommand{\xxxSubParagraphStar}[1]{\oldsubparagraph*{#1}\mbox{}}
  \newcommand{\xxxSubParagraphNoStar}[1]{\oldsubparagraph{#1}\mbox{}}
\fi
\makeatother

\usepackage{color}
\usepackage{fancyvrb}
\newcommand{\VerbBar}{|}
\newcommand{\VERB}{\Verb[commandchars=\\\{\}]}
\DefineVerbatimEnvironment{Highlighting}{Verbatim}{commandchars=\\\{\}}
% Add ',fontsize=\small' for more characters per line
\usepackage{framed}
\definecolor{shadecolor}{RGB}{241,243,245}
\newenvironment{Shaded}{\begin{snugshade}}{\end{snugshade}}
\newcommand{\AlertTok}[1]{\textcolor[rgb]{0.68,0.00,0.00}{#1}}
\newcommand{\AnnotationTok}[1]{\textcolor[rgb]{0.37,0.37,0.37}{#1}}
\newcommand{\AttributeTok}[1]{\textcolor[rgb]{0.40,0.45,0.13}{#1}}
\newcommand{\BaseNTok}[1]{\textcolor[rgb]{0.68,0.00,0.00}{#1}}
\newcommand{\BuiltInTok}[1]{\textcolor[rgb]{0.00,0.23,0.31}{#1}}
\newcommand{\CharTok}[1]{\textcolor[rgb]{0.13,0.47,0.30}{#1}}
\newcommand{\CommentTok}[1]{\textcolor[rgb]{0.37,0.37,0.37}{#1}}
\newcommand{\CommentVarTok}[1]{\textcolor[rgb]{0.37,0.37,0.37}{\textit{#1}}}
\newcommand{\ConstantTok}[1]{\textcolor[rgb]{0.56,0.35,0.01}{#1}}
\newcommand{\ControlFlowTok}[1]{\textcolor[rgb]{0.00,0.23,0.31}{\textbf{#1}}}
\newcommand{\DataTypeTok}[1]{\textcolor[rgb]{0.68,0.00,0.00}{#1}}
\newcommand{\DecValTok}[1]{\textcolor[rgb]{0.68,0.00,0.00}{#1}}
\newcommand{\DocumentationTok}[1]{\textcolor[rgb]{0.37,0.37,0.37}{\textit{#1}}}
\newcommand{\ErrorTok}[1]{\textcolor[rgb]{0.68,0.00,0.00}{#1}}
\newcommand{\ExtensionTok}[1]{\textcolor[rgb]{0.00,0.23,0.31}{#1}}
\newcommand{\FloatTok}[1]{\textcolor[rgb]{0.68,0.00,0.00}{#1}}
\newcommand{\FunctionTok}[1]{\textcolor[rgb]{0.28,0.35,0.67}{#1}}
\newcommand{\ImportTok}[1]{\textcolor[rgb]{0.00,0.46,0.62}{#1}}
\newcommand{\InformationTok}[1]{\textcolor[rgb]{0.37,0.37,0.37}{#1}}
\newcommand{\KeywordTok}[1]{\textcolor[rgb]{0.00,0.23,0.31}{\textbf{#1}}}
\newcommand{\NormalTok}[1]{\textcolor[rgb]{0.00,0.23,0.31}{#1}}
\newcommand{\OperatorTok}[1]{\textcolor[rgb]{0.37,0.37,0.37}{#1}}
\newcommand{\OtherTok}[1]{\textcolor[rgb]{0.00,0.23,0.31}{#1}}
\newcommand{\PreprocessorTok}[1]{\textcolor[rgb]{0.68,0.00,0.00}{#1}}
\newcommand{\RegionMarkerTok}[1]{\textcolor[rgb]{0.00,0.23,0.31}{#1}}
\newcommand{\SpecialCharTok}[1]{\textcolor[rgb]{0.37,0.37,0.37}{#1}}
\newcommand{\SpecialStringTok}[1]{\textcolor[rgb]{0.13,0.47,0.30}{#1}}
\newcommand{\StringTok}[1]{\textcolor[rgb]{0.13,0.47,0.30}{#1}}
\newcommand{\VariableTok}[1]{\textcolor[rgb]{0.07,0.07,0.07}{#1}}
\newcommand{\VerbatimStringTok}[1]{\textcolor[rgb]{0.13,0.47,0.30}{#1}}
\newcommand{\WarningTok}[1]{\textcolor[rgb]{0.37,0.37,0.37}{\textit{#1}}}

\usepackage{longtable,booktabs,array}
\usepackage{calc} % for calculating minipage widths
% Correct order of tables after \paragraph or \subparagraph
\usepackage{etoolbox}
\makeatletter
\patchcmd\longtable{\par}{\if@noskipsec\mbox{}\fi\par}{}{}
\makeatother
% Allow footnotes in longtable head/foot
\IfFileExists{footnotehyper.sty}{\usepackage{footnotehyper}}{\usepackage{footnote}}
\makesavenoteenv{longtable}
\usepackage{graphicx}
\makeatletter
\newsavebox\pandoc@box
\newcommand*\pandocbounded[1]{% scales image to fit in text height/width
  \sbox\pandoc@box{#1}%
  \Gscale@div\@tempa{\textheight}{\dimexpr\ht\pandoc@box+\dp\pandoc@box\relax}%
  \Gscale@div\@tempb{\linewidth}{\wd\pandoc@box}%
  \ifdim\@tempb\p@<\@tempa\p@\let\@tempa\@tempb\fi% select the smaller of both
  \ifdim\@tempa\p@<\p@\scalebox{\@tempa}{\usebox\pandoc@box}%
  \else\usebox{\pandoc@box}%
  \fi%
}
% Set default figure placement to htbp
\def\fps@figure{htbp}
\makeatother



\ifLuaTeX
\usepackage[bidi=basic]{babel}
\else
\usepackage[bidi=default]{babel}
\fi
\ifPDFTeX
\else
\babelfont{rm}[]{Times New Roman}
\fi
% get rid of language-specific shorthands (see #6817):
\let\LanguageShortHands\languageshorthands
\def\languageshorthands#1{}


\setlength{\emergencystretch}{3em} % prevent overfull lines

\providecommand{\tightlist}{%
  \setlength{\itemsep}{0pt}\setlength{\parskip}{0pt}}


\usepackage[authordate]{biblatex-chicago}
\addbibresource{bibliography/references.bib}


\usepackage{titlesec}
\usepackage{setspace}
\usepackage{sectsty}

\titleformat{\chapter}[hang]
  {\centering\bfseries\LARGE} % Formatting for the title
  {Chapitre \thechapter \ -- } % Prefix for the chapter title
  {0cm} % Space between the prefix and the title
  {} % Additional formatting for the chapter title itself

\titleformat{\section}[hang]
  {\bfseries\bfseries} % Formatting for the title
  {\thesection \ } % Prefix for the chapter title
  {0cm} % Space between the prefix and the title
  {} % Additional formatting for the chapter title itself

\titleformat{\subsection}[hang]
  {\bfseries} % Formatting for the title
  {\thesubsection \ } % Prefix for the chapter title
  {0cm} % Space between the prefix and the title
  {}

\titleformat{\subsubsection}[hang]
  {} % Formatting for the title
  {\thesubsubsection \ } % Prefix for the chapter title
  {0cm} % Space between the prefix and the title
  {}

\titlespacing*{\chapter}{0pt}{-1cm}{12pt}
\titlespacing*{\section}{0pt}{0pt}{0pt}
\titlespacing*{\subsection}{1.25cm}{0pt}{0pt}

% \chapterfont{\fontsize{18}{0}\selectfont}
\sectionfont{\fontsize{16}{0}\selectfont} % Change section font size
\subsectionfont{\fontsize{14}{0}\selectfont} % Change subsection font size
\subsubsectionfont{\fontsize{12}{0}\selectfont} % Change subsubsection font size
\usepackage{float}
\usepackage{tabularray}
\usepackage[normalem]{ulem}
\usepackage{graphicx}
\usepackage{rotating}
\UseTblrLibrary{booktabs}
\UseTblrLibrary{siunitx}
\NewTableCommand{\tinytableDefineColor}[3]{\definecolor{#1}{#2}{#3}}
\newcommand{\tinytableTabularrayUnderline}[1]{\underline{#1}}
\newcommand{\tinytableTabularrayStrikeout}[1]{\sout{#1}}
\usepackage{pdflscape}
\usepackage{csquotes}
\usepackage{mathtools}
\usepackage{amsmath} % For advanced math symbols
\usepackage{amssymb} % For additional symbols
\usepackage{tikz}
\usetikzlibrary{mindmap}
\usepackage{xeCJK}
\makeatletter
\@ifpackageloaded{bookmark}{}{\usepackage{bookmark}}
\makeatother
\makeatletter
\@ifpackageloaded{caption}{}{\usepackage{caption}}
\AtBeginDocument{%
\ifdefined\contentsname
  \renewcommand*\contentsname{Table des matières}
\else
  \newcommand\contentsname{Table des matières}
\fi
\ifdefined\listfigurename
  \renewcommand*\listfigurename{Liste des Figures}
\else
  \newcommand\listfigurename{Liste des Figures}
\fi
\ifdefined\listtablename
  \renewcommand*\listtablename{Liste des Tables}
\else
  \newcommand\listtablename{Liste des Tables}
\fi
\ifdefined\figurename
  \renewcommand*\figurename{Figure}
\else
  \newcommand\figurename{Figure}
\fi
\ifdefined\tablename
  \renewcommand*\tablename{Tableau}
\else
  \newcommand\tablename{Tableau}
\fi
}
\@ifpackageloaded{float}{}{\usepackage{float}}
\floatstyle{ruled}
\@ifundefined{c@chapter}{\newfloat{codelisting}{h}{lop}}{\newfloat{codelisting}{h}{lop}[chapter]}
\floatname{codelisting}{Listing}
\newcommand*\listoflistings{\listof{codelisting}{Liste des Listings}}
\makeatother
\makeatletter
\makeatother
\makeatletter
\@ifpackageloaded{caption}{}{\usepackage{caption}}
\@ifpackageloaded{subcaption}{}{\usepackage{subcaption}}
\makeatother
\usepackage{bookmark}
\IfFileExists{xurl.sty}{\usepackage{xurl}}{} % add URL line breaks if available
\urlstyle{same}
\hypersetup{
  pdftitle={Titre du mémoire ou de la thèse},
  pdfauthor={Auteur/Autrice},
  pdflang={fr-FR},
  pdfkeywords={Keyword 1, keyword 2, etc.},
  colorlinks=true,
  linkcolor={black},
  filecolor={Maroon},
  citecolor={black},
  urlcolor={black},
  pdfcreator={LaTeX via pandoc}}


\title{Titre du mémoire ou de la thèse}
\author{Auteur/Autrice}
\date{Janvier 2026}
\begin{document}
\cleardoublepage
\thispagestyle{empty}
{\centering
Université de Montréal \\[2.5cm]
{\itshape Titre du mémoire ou de la thèse \par}
\vspace{2.5cm}
{Par \\ Auteur/Autrice \par} %%%% AJOUTER SON PRÉNOM ET SON NOM DE FAMILLE 
\vspace{2.5cm}
{Département de science politique, Université de Montréal \\
	Faculté des arts et des sciences \par}
\vspace{2.5cm}

{Thèse présentée en vue de l'obtention \\
	du grade de Philosophiae Doctor (Ph.D.) en science politique \par}
\vspace{2.5cm}
{ Janvier 2026 \par} %%%%% AJOUTER LA DATE
\vspace{2.5cm}
{\textcopyright{Auteur/Autrice} } %%%%% AJOUTER LES INFORMATIONS POUR LE COPYRIGHT 
\clearpage
}

\thispagestyle{empty}
{\centering
Université de Montréal \\
Faculté des Arts et des Sciences, Département de Science Politique \\
\hrulefill \\[2.5cm]
{\itshape Cette thèse intitulée \\
\textbf{ Titre du mémoire ou de la
thèse } \par}  %%%%%% AJOUTER LE TITRE
\vspace{2.5cm}
Présenté par \\
{\textbf{Auteur/Autrice} \par}  %%%%%%% AJOUTER VOTRE NOM ET VOTRE PRÉNOM
\vspace{2.5cm}
{\itshape A été évalué par un jury composé des personnes suivantes :} \\[1cm]
}
\cleardoublepage
\thispagestyle{plain}
\section*{\centering{Résumé}}
\begin{spacing}{1}
	Ajouter un résumé en français ici.
\end{spacing}
\vspace{1cm}

\noindent \textbf{Mots-clés :} Mot clé 1, mot clé 2, etc. \clearpage

\section*{\centering{Abstract}}
\begin{spacing}{1}
	Add the English abstract here.
\end{spacing}
\vspace{1cm}

\noindent \textbf{Keywords :} Keyword 1, keyword 2, etc. \clearpage

\renewcommand*\contentsname{Table des matières}
{
\hypersetup{linkcolor=}
\setcounter{tocdepth}{3}
\tableofcontents
}
\listoffigures
\listoftables

\setstretch{1.5}
\bookmarksetup{startatroot}

\chapter*{Remerciements}\label{remerciements}
\addcontentsline{toc}{chapter}{Remerciements}

\markboth{Remerciements}{Remerciements}

\bookmarksetup{startatroot}

\chapter{Introduction}\label{introduction}

Ajout d'une référence afin d'illustrer le format complet du livre
\autocite{arel-bundock2021}.

\bookmarksetup{startatroot}

\chapter{Revue de littérature et
théorie}\label{revue-de-littuxe9rature-et-thuxe9orie}

\bookmarksetup{startatroot}

\chapter{Méthodologie}\label{muxe9thodologie}

\bookmarksetup{startatroot}

\chapter{Analyses}\label{analyses}

Quarto permet un intégration facile avec \texttt{R}. Ainsi, ajouter des
tableaux et des graphiques se fait facilement. La
figure~\ref{fig-super-graphique} presésente un example. Le code du
graphique a été ajouté par défault ici, mais il est désactivé
globalement dans le document
\texttt{\_extensions/these-memoire-udem/\_extension.yml} avec l'option
\texttt{echo:\ false}.

\begin{Shaded}
\begin{Highlighting}[]
\FunctionTok{library}\NormalTok{(palmerpenguins)}
\FunctionTok{library}\NormalTok{(tidyverse)}
\NormalTok{dat }\OtherTok{\textless{}{-}}\NormalTok{ penguins}
\FunctionTok{ggplot}\NormalTok{(}
\NormalTok{  dat,}
  \FunctionTok{aes}\NormalTok{(bill\_length\_mm, body\_mass\_g, }\AttributeTok{color =}\NormalTok{ species, }\AttributeTok{shape =}\NormalTok{ species)}
\NormalTok{) }\SpecialCharTok{+}
  \FunctionTok{geom\_point}\NormalTok{() }\SpecialCharTok{+}
  \FunctionTok{geom\_smooth}\NormalTok{(}\AttributeTok{method =}\NormalTok{ loess) }\SpecialCharTok{+}
  \FunctionTok{theme\_bw}\NormalTok{()}
\end{Highlighting}
\end{Shaded}

\begin{figure}[H]

\centering{

\pandocbounded{\includegraphics[keepaspectratio]{analyses_files/figure-pdf/fig-super-graphique-1.pdf}}

}

\caption{\label{fig-super-graphique}Super graphique}

\end{figure}%

Cela peut aussi être fait à l'aide d'un document source qui est situé
dans un autre fichier.

\begin{Shaded}
\begin{Highlighting}[]
\FunctionTok{source}\NormalTok{(}\StringTok{"code/analyses.R"}\NormalTok{)}
\NormalTok{summary\_data}
\end{Highlighting}
\end{Shaded}

\begin{table}

\caption{\label{tbl-resume}Super tableau}

\centering{

\centering
\begin{tblr}[         %% tabularray outer open
]                     %% tabularray outer close
{                     %% tabularray inner open
colspec={Q[]Q[]Q[]Q[]Q[]Q[]Q[]Q[]Q[]},
column{1-9}={}{halign=l,},
}                     %% tabularray inner close
\toprule
& Unique & Missing Pct. & Mean & SD & Min & Median & Max & Histogram \\ \midrule %% TinyTableHeader
mpg & 25 & 0 & \num{20.1} & \num{6.0} & \num{10.4} & \num{19.2} & \num{33.9} & \includegraphics[height=1em]{tinytable_assets/id58y08n51prpc508h58q6.png} \\
cyl & 3 & 0 & \num{6.2} & \num{1.8} & \num{4.0} & \num{6.0} & \num{8.0} & \includegraphics[height=1em]{tinytable_assets/id55fogytepfyb8tvvxkcd.png} \\
disp & 27 & 0 & \num{230.7} & \num{123.9} & \num{71.1} & \num{196.3} & \num{472.0} & \includegraphics[height=1em]{tinytable_assets/id2i76mt6872coyrc35j6u.png} \\
hp & 22 & 0 & \num{146.7} & \num{68.6} & \num{52.0} & \num{123.0} & \num{335.0} & \includegraphics[height=1em]{tinytable_assets/idld8iwuiu2czsdeckjjr6.png} \\
drat & 22 & 0 & \num{3.6} & \num{0.5} & \num{2.8} & \num{3.7} & \num{4.9} & \includegraphics[height=1em]{tinytable_assets/idak4l1wyedme92ftzs5yk.png} \\
wt & 29 & 0 & \num{3.2} & \num{1.0} & \num{1.5} & \num{3.3} & \num{5.4} & \includegraphics[height=1em]{tinytable_assets/ido5ztcymjxofy8rswdhoj.png} \\
qsec & 30 & 0 & \num{17.8} & \num{1.8} & \num{14.5} & \num{17.7} & \num{22.9} & \includegraphics[height=1em]{tinytable_assets/id9e8wctqes999bkvvsg8q.png} \\
vs & 2 & 0 & \num{0.4} & \num{0.5} & \num{0.0} & \num{0.0} & \num{1.0} & \includegraphics[height=1em]{tinytable_assets/idxgs0l3oorm0w1fud3j81.png} \\
am & 2 & 0 & \num{0.4} & \num{0.5} & \num{0.0} & \num{0.0} & \num{1.0} & \includegraphics[height=1em]{tinytable_assets/idsfz4o4d019yv3n2q37i6.png} \\
gear & 3 & 0 & \num{3.7} & \num{0.7} & \num{3.0} & \num{4.0} & \num{5.0} & \includegraphics[height=1em]{tinytable_assets/idpu078qz63cb6oq9pbdu9.png} \\
carb & 6 & 0 & \num{2.8} & \num{1.6} & \num{1.0} & \num{2.0} & \num{8.0} & \includegraphics[height=1em]{tinytable_assets/idy0kcf0l63q761vj6ot11.png} \\
\bottomrule
\end{tblr}

}

\end{table}%

\bookmarksetup{startatroot}

\chapter{Discussion}\label{discussion}

\bookmarksetup{startatroot}

\chapter{Conclusion}\label{conclusion}

\bookmarksetup{startatroot}

\chapter*{Bibliographie}\label{bibliographie}
\addcontentsline{toc}{chapter}{Bibliographie}

\markboth{Bibliographie}{Bibliographie}

\printbibliography[heading=none]

\cleardoublepage
\phantomsection
\addcontentsline{toc}{part}{Appendices}
\appendix

\chapter*{Annexes}\label{annexes}
\addcontentsline{toc}{chapter}{Annexes}

\markboth{Annexes}{Annexes}





\end{document}
